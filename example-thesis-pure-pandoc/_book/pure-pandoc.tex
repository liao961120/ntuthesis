  \documentclass[oneside]{book}
%\documentclass[]{book}

\usepackage{lmodern}
\usepackage{setspace}
\setstretch{1.5}
\usepackage{amssymb,amsmath}
\usepackage{ifxetex,ifluatex}
\usepackage{fixltx2e} % provides \textsubscript
\ifnum 0\ifxetex 1\fi\ifluatex 1\fi=0 % if pdftex
  \usepackage[T1]{fontenc}
  \usepackage[utf8]{inputenc}
\else % if luatex or xelatex
  \ifxetex
    \usepackage{xltxtra,xunicode}
  \else
    \usepackage{fontspec}
  \fi
  \defaultfontfeatures{Ligatures=TeX,Scale=MatchLowercase}


\fi
% use upquote if available, for straight quotes in verbatim environments
\IfFileExists{upquote.sty}{\usepackage{upquote}}{}
% use microtype if available
\IfFileExists{microtype.sty}{%
\usepackage{microtype}
\UseMicrotypeSet[protrusion]{basicmath} % disable protrusion for tt fonts
}{}
\usepackage[a4paper, left=1.18in, right=1.18in, top=1.18in, bottom=0.787in]{geometry}
\usepackage[unicode=true]{hyperref}
\hypersetup{
            pdftitle={臺大論文模板},
            pdfauthor={廖永賦},
            pdfborder={0 0 0},
            breaklinks=true}
\urlstyle{same}  % don't use monospace font for urls
\usepackage{color}
\usepackage{fancyvrb}
\newcommand{\VerbBar}{|}
\newcommand{\VERB}{\Verb[commandchars=\\\{\}]}
\DefineVerbatimEnvironment{Highlighting}{Verbatim}{commandchars=\\\{\}}
% Add ',fontsize=\small' for more characters per line
\usepackage{framed}
\definecolor{shadecolor}{RGB}{248,248,248}
\newenvironment{Shaded}{\begin{snugshade}}{\end{snugshade}}
\newcommand{\AlertTok}[1]{\textcolor[rgb]{0.94,0.16,0.16}{#1}}
\newcommand{\AnnotationTok}[1]{\textcolor[rgb]{0.56,0.35,0.01}{\textbf{\textit{#1}}}}
\newcommand{\AttributeTok}[1]{\textcolor[rgb]{0.77,0.63,0.00}{#1}}
\newcommand{\BaseNTok}[1]{\textcolor[rgb]{0.00,0.00,0.81}{#1}}
\newcommand{\BuiltInTok}[1]{#1}
\newcommand{\CharTok}[1]{\textcolor[rgb]{0.31,0.60,0.02}{#1}}
\newcommand{\CommentTok}[1]{\textcolor[rgb]{0.56,0.35,0.01}{\textit{#1}}}
\newcommand{\CommentVarTok}[1]{\textcolor[rgb]{0.56,0.35,0.01}{\textbf{\textit{#1}}}}
\newcommand{\ConstantTok}[1]{\textcolor[rgb]{0.00,0.00,0.00}{#1}}
\newcommand{\ControlFlowTok}[1]{\textcolor[rgb]{0.13,0.29,0.53}{\textbf{#1}}}
\newcommand{\DataTypeTok}[1]{\textcolor[rgb]{0.13,0.29,0.53}{#1}}
\newcommand{\DecValTok}[1]{\textcolor[rgb]{0.00,0.00,0.81}{#1}}
\newcommand{\DocumentationTok}[1]{\textcolor[rgb]{0.56,0.35,0.01}{\textbf{\textit{#1}}}}
\newcommand{\ErrorTok}[1]{\textcolor[rgb]{0.64,0.00,0.00}{\textbf{#1}}}
\newcommand{\ExtensionTok}[1]{#1}
\newcommand{\FloatTok}[1]{\textcolor[rgb]{0.00,0.00,0.81}{#1}}
\newcommand{\FunctionTok}[1]{\textcolor[rgb]{0.00,0.00,0.00}{#1}}
\newcommand{\ImportTok}[1]{#1}
\newcommand{\InformationTok}[1]{\textcolor[rgb]{0.56,0.35,0.01}{\textbf{\textit{#1}}}}
\newcommand{\KeywordTok}[1]{\textcolor[rgb]{0.13,0.29,0.53}{\textbf{#1}}}
\newcommand{\NormalTok}[1]{#1}
\newcommand{\OperatorTok}[1]{\textcolor[rgb]{0.81,0.36,0.00}{\textbf{#1}}}
\newcommand{\OtherTok}[1]{\textcolor[rgb]{0.56,0.35,0.01}{#1}}
\newcommand{\PreprocessorTok}[1]{\textcolor[rgb]{0.56,0.35,0.01}{\textit{#1}}}
\newcommand{\RegionMarkerTok}[1]{#1}
\newcommand{\SpecialCharTok}[1]{\textcolor[rgb]{0.00,0.00,0.00}{#1}}
\newcommand{\SpecialStringTok}[1]{\textcolor[rgb]{0.31,0.60,0.02}{#1}}
\newcommand{\StringTok}[1]{\textcolor[rgb]{0.31,0.60,0.02}{#1}}
\newcommand{\VariableTok}[1]{\textcolor[rgb]{0.00,0.00,0.00}{#1}}
\newcommand{\VerbatimStringTok}[1]{\textcolor[rgb]{0.31,0.60,0.02}{#1}}
\newcommand{\WarningTok}[1]{\textcolor[rgb]{0.56,0.35,0.01}{\textbf{\textit{#1}}}}
\usepackage{longtable,booktabs}
% Fix footnotes in tables (requires footnote package)
\IfFileExists{footnote.sty}{\usepackage{footnote}\makesavenoteenv{long table}}{}
\usepackage{graphicx,grffile}
\makeatletter
\def\maxwidth{\ifdim\Gin@nat@width>\linewidth\linewidth\else\Gin@nat@width\fi}
\def\maxheight{\ifdim\Gin@nat@height>\textheight\textheight\else\Gin@nat@height\fi}
\makeatother
% Scale images if necessary, so that they will not overflow the page
% margins by default, and it is still possible to overwrite the defaults
% using explicit options in \includegraphics[width, height, ...]{}
\setkeys{Gin}{width=\maxwidth,height=\maxheight,keepaspectratio}
\let\oldhref=\href
% Make links footnotes instead of hotlinks:
\renewcommand{\href}[2]{#2\footnote{\url{#1}}}
\IfFileExists{parskip.sty}{%
\usepackage{parskip}
}{% else
\setlength{\parindent}{0pt}
\setlength{\parskip}{6pt plus 2pt minus 1pt}
}
\setlength{\emergencystretch}{3em}  % prevent overfull lines
\providecommand{\tightlist}{%
  \setlength{\itemsep}{0pt}\setlength{\parskip}{0pt}}
\setcounter{secnumdepth}{5}

% set default figure placement to htbp
\makeatletter
\def\fps@figure{htbp}
\makeatother

\usepackage{pdfpages}
\usepackage{titlesec}
\usepackage{titletoc}
\usepackage{booktabs}
\usepackage{apptools}
\usepackage{float}
\usepackage[section]{placeins}

\usepackage[heading, fontset = none]{ctex}
\ctexset{appendix/name={\appendixname\space}}

\usepackage[fontsize=12pt]{scrextend}

% Eng font-family
\setmainfont[
  Path=latex/,
  BoldFont={TimesNewRomanBold},
  ItalicFont={TimesNewRomanItalic},
  BoldItalicFont={TimesNewRomanBoldItalic}
]{TimesNewRoman}

% Trad Ch font-family
\setCJKmainfont[Path=latex/,AutoFakeBold=2.5,AutoFakeSlant=.3]{kaiti}
\setCJKmonofont[Path=latex/]{NotoSansMonoCJKtc}

% Special font: IPA
\newfontfamily{\ipa}[Path=latex/,AutoFakeBold=2.5,AutoFakeSlant=.3]{IPAfont} % Font for IPA symbols
\DeclareTextFontCommand{\ipatext}{\ipa}


%中文自動換行
\XeTeXlinebreaklocale "zh"
%文字的彈性間距
\XeTeXlinebreakskip = 0pt plus 1pt



\renewcommand{\figurename}{圖}
\renewcommand{\tablename}{表}
\renewcommand{\contentsname}{目錄}
\renewcommand{\listfigurename}{圖目錄}
\renewcommand{\listtablename}{表目錄}
\renewcommand{\appendixname}{附錄}
%\renewcommand{\bibname}{參考資料}

% deal with nuts floating figures
\renewcommand{\textfraction}{0.05}
\renewcommand{\topfraction}{0.8}
\renewcommand{\bottomfraction}{0.8}
\renewcommand{\floatpagefraction}{0.75}

\title{臺大論文模板}
\author{廖永賦}
\date{May 24, 2019}


\usepackage{fontspec}
%使用xeCJK,其他的還有CJK或是xCJK
\usepackage{xeCJK}
\usepackage{bm}

% Set the default fonts
% See https://tug.org/pipermail/xetex/2011-March/020226.html for fontspec
% % \setmainfont[
%   Path=latex/,
%   BoldFont={TimesNewRomanBold.ttf},
%   ItalicFont={TimesNewRomanItalic.ttf},
%   BoldItalicFont={TimesNewRomanBoldItalic.ttf}
% ]{TimesNewRoman.ttf}
% 
% %     \setCJKmainfont[Path=latex/,AutoFakeBold=2.5,AutoFakeSlant=.3]{kaiti}
%     \setCJKmonofont[Path=latex/]{NotoSansMonoCJKtc}
% 


% IPA support (Works with linguisticsdown)
% 

\usepackage{xcolor}
\usepackage{transparent}

\usepackage{tikz}
\usepackage[printwatermark]{xwatermark}
\newsavebox\mybox
\savebox\mybox{\tikz[]\node[opacity=0.2]{\includegraphics{watermark.png}};}
\newwatermark*[
  allpages,
  %angle=45,
  scale=0.18,
  xpos=6.3725cm,
  ypos=10.8225cm
]{\usebox\mybox}






\begin{document}


\includepdf[pages={1}, scale=1]{front_matter/front_matter.pdf}

\clearpage
\pagenumbering{roman}

\phantomsection
\addcontentsline{toc}{chapter}{口試委員會審定書}
\includepdf[pages={1}, scale=1]{certification-scan.pdf}


\phantomsection
\chapter*{誌謝}
非常感謝網路上各個默默耕耘開發 open source 專案的大大們。
非常感謝網路上各個默默耕耘開發 open source 專案的大大們。
非常感謝網路上各個默默耕耘開發 open source 專案的大大們。

沒有這些既存的資源,這份模板是不可能出現的。沒有這些既存的資源,這份模板是不可能出現的。沒有這些既存的資源,這份模板是不可能出現的。沒有這些既存的資源,這份模板是不可能出現的。
\addcontentsline{toc}{chapter}{誌謝}


\phantomsection
\chapter*{摘要}
摘要\textbf{第 5 行}開始而且不能是空行。摘要\textbf{第 5 行}開始而且不能是空行。摘要\textbf{第 5 行}開始而且不能是空行。摘要\textbf{第 5 行}開始而且不能是空行。

新段落要在前面空一行。新段落要在前面空一行。新段落要在前面空一行。新段落要在前面空一行。新段落要在前面空一行。
\bigbreak

\noindent
\textbf{關鍵字:} 第二行開始、R Markdown、Bookdown、可重製研究
\addcontentsline{toc}{chapter}{中文摘要}

\phantomsection
\chapter*{Abstract}
The first line of the abstract starts on \textbf{line 5} and must not be blank. The first line of the abstract starts on \textbf{line 5} and must not be blank. The first line of the abstract starts on \textbf{line 5} and must not be blank.

A new paragraph of the abstract. A new paragraph of the abstract. A new paragraph of the abstract. A new paragraph of the abstract. A new paragraph of the abstract. A new paragraph of the abstract.
\bigbreak

\noindent
\textbf{Keywords:} Line 2, R Markdown, Bookdown, Reproducible Research
\addcontentsline{toc}{chapter}{英文摘要}


{
\setcounter{tocdepth}{1}
\tableofcontents
%\phantomsection
%\addcontentsline{toc}{chapter}{\contentsname}
}

\newpage

\listoftables
\phantomsection
\addcontentsline{toc}{chapter}{\listtablename}
\newpage

\listoffigures
\phantomsection
\addcontentsline{toc}{chapter}{\listfigurename}
\newpage

% Set independent linestretch for code chunks
% Doesn't work in pure pandoc
%\let\oldShaded=\Shaded
%\let\endoldShaded=\endShaded
%\renewenvironment{Shaded}{
%  %    \begin{spacing}{1}\begin{oldShaded}
%  %  }
%  {
%  \end{oldShaded}
%  \end{spacing}
%  }

\clearpage
\pagenumbering{arabic}

\hypertarget{write-thesis}{%
\chapter{論文撰寫}\label{write-thesis}}

\hypertarget{dir-structure}{%
\section{檔案結構}\label{dir-structure}}

執行以下指令後(詳見 \ref{import-template})

\begin{Shaded}
\begin{Highlighting}[]
\NormalTok{ntuthesis}\OperatorTok{::}\KeywordTok{import_template}\NormalTok{(}\StringTok{"project_name"}\NormalTok{)}
\end{Highlighting}
\end{Shaded}

即會匯入論文模板。以下是論文模板的檔案結構(已簡化):

\begin{Shaded}
\begin{Highlighting}[]
\NormalTok{├── project_name.Rmd     }\CommentTok{# Useless, please delete it}
\NormalTok{|}
\NormalTok{├── R/                   }\CommentTok{# code chunk root dir, put R scripts and data here}
\NormalTok{├── figs/                }\CommentTok{# Put figures to include in the thesis here}
\NormalTok{|}
\NormalTok{├── index.Rmd            }\CommentTok{# Book Layout (font, watermark, biblio, ...)}
\NormalTok{├── _acknowledge.Rmd     }\CommentTok{# acknowledgement}
\NormalTok{├── _abstract-en.Rmd     }\CommentTok{# abstract}
\NormalTok{├── _abstract-zh.Rmd     }\CommentTok{# Same as above, but in Chinese}
\NormalTok{|}
\NormalTok{├── 01-intro.Rmd         }\CommentTok{# Chapter 1 content}
\NormalTok{├── 02-literature.Rmd    }\CommentTok{# Chapter 2 content}
\NormalTok{├── 03-method.Rmd        }\CommentTok{# Chapter 3 content}
\NormalTok{├── 80-appx-help.Rmd     }\CommentTok{# Appendix Content}
\NormalTok{├── 99-references.Rmd    }\CommentTok{# Edit "References" Title}
\NormalTok{├── ref.bib              }\CommentTok{# References}
\NormalTok{├── cite-style.csl       }\CommentTok{# Citation style}
\NormalTok{|}
\NormalTok{├── _bookdown.yml        }\CommentTok{# label names in gitbook; Rmd files order}
\NormalTok{├── _output.yml          }\CommentTok{# preamble, pandoc args, cite-pkg}
\NormalTok{|}
\NormalTok{├── watermark.pdf        }\CommentTok{# 臺大浮水印 (PDF 右上角)}
\NormalTok{├── _person-info.yml      }\CommentTok{# Info to generate front matter}
\NormalTok{├── certification-scan.pdf  }\CommentTok{# 已簽名'口試委員審查書'}
\NormalTok{└── front_matter}
\NormalTok{    └── certification.pdf   }\CommentTok{# 空白'口試委員審查書'}
\end{Highlighting}
\end{Shaded}

\hypertarget{index-rmd}{%
\section{\texorpdfstring{\texttt{index.Rmd}}{index.Rmd}}\label{index-rmd}}

\texttt{index.Rmd} 是設定論文內文格式的地方,包含 yaml 以及 R setup code chunk。此模板將 code chunk 預設的 working directory 改成 \texttt{R/}\footnote{預設是 Rmd 檔所在的位置。},如此較符合一般寫 Rscript 的邏輯\footnote{例如,使用相對路徑匯入資料時,一般會以 Rscript 所在的位置作為基準。}。若要更改此設定,至 setup code chunk 更改 \texttt{knitr::opts\_knit\$set(root.dir=\textquotesingle{}R\textquotesingle{})}。

\hypertarget{write-lang}{%
\section{撰寫語言}\label{write-lang}}

若使用\textbf{英文}撰寫論文,需修改 \texttt{\_output.yml}、\texttt{\_bookdown.yml} 這兩個檔案的內容。

\hypertarget{output.yml}{%
\subsection{\texorpdfstring{\texttt{\_output.yml}}{\_output.yml}}\label{output.yml}}

將 \texttt{in\_header:\ latex/preamble-zh.tex} 改為 \texttt{in\_header:\ latex/preamble-en.tex}:

\begin{Shaded}
\begin{Highlighting}[]
\FunctionTok{bookdown:}\AttributeTok{:pdf_book:}
  \FunctionTok{includes:}
    \FunctionTok{in_header:}\AttributeTok{ latex/preamble-en.tex}
\end{Highlighting}
\end{Shaded}

\hypertarget{bookdown.yml}{%
\subsection{\texorpdfstring{\texttt{\_bookdown.yml}}{\_bookdown.yml}}\label{bookdown.yml}}

\texttt{\_bookdown.yml} 中,可以對標籤的名稱進行定義。這裡的設定與 PDF 輸出無關,只與 gitbook 輸出格式有關。因此,若無需使用 gitbook 輸出,可忽略此段。

此外,\texttt{\_bookdown.yml} 亦可設定 Rmd 檔在輸出文件中的順序。若無設定,就會依序檔名排序\footnote{此模板即未進行設定,因此第一章的內容寫在 \texttt{01-xxx.Rmd} 就會自動排在第一。而若檔名以底線開頭(\texttt{\_})則會被忽略。更多內容詳見 \href{https://bookdown.org/yihui/bookdown/usage.html}{bookdown}。}。

在以下設定中,可使 gitbook 輸出的章節(順序)與 PDF 不同。

\begin{Shaded}
\begin{Highlighting}[]
\FunctionTok{rmd_files:}
  \FunctionTok{html:}\AttributeTok{ }\KeywordTok{[}\StringTok{"index.Rmd"}\KeywordTok{,} \StringTok{"abstract.Rmd"}\KeywordTok{,} \StringTok{"intro.Rmd"}\KeywordTok{]}
  \FunctionTok{latex:}\AttributeTok{ }\KeywordTok{[}\StringTok{"abstract.Rmd"}\KeywordTok{,} \StringTok{"intro.Rmd"}\KeywordTok{]}
\end{Highlighting}
\end{Shaded}

\hypertarget{bib-cite}{%
\section{文獻引用}\label{bib-cite}}

R Markdown 在文章中插入引用文獻的功能承繼 Pandoc。完整的使用見 \href{https://rmarkdown.rstudio.com/authoring_bibliographies_and_citations.html}{R Markdown 官方說明} 。

此模板目前產生文獻格式的方法是依靠 \href{https://github.com/jgm/pandoc-citeproc}{Pandoc citeproc},因此,文獻格式是依據 \texttt{cite-style.csl}\footnote{此模板提供的 \texttt{cite-style.csl} 是 APA 英文第六版。此外,\url{http://blog.pulipuli.info/2011/05/zoteroapa.html} 亦有提供 APA 中文版的引用格式。需注意的是 Pandoc \textbf{不支援雙語 csl} (\url{http://blog.pulipuli.info/2014/08/zoteroapa-zotero-citation-style-apa.html})。} 產生的。使用者可至 \href{https://www.zotero.org/styles}{Zotero Style Repository} 下載所需的 csl 檔並覆蓋專案資料夾中的 \texttt{cite-style.csl}。

\hypertarget{ref-bib}{%
\subsection{\texorpdfstring{\texttt{ref.bib}}{ref.bib}}\label{ref-bib}}

\texttt{.bib} 檔的產生方式可以由 Endnote, Zotero, JabRef 等書目管理軟體匯出。匯出後,將檔名命名為 \texttt{ref.bib} 放在專案資料夾\footnote{或是可以自訂檔名,並到 \texttt{index.Rmd} yaml 中的 \texttt{bibliography:\ ref.bib} 更改 \texttt{ref.bib} 檔名。此外,亦可使用多個 \texttt{.bib} 檔:\texttt{bibliography:\ {[}ref1.bib,\ ref2.bib,\ ref3.bib{]}}。}。

\texttt{.bib} 內的一篇引用資料會類似:

\begin{Shaded}
\begin{Highlighting}[]
\VariableTok{@article}\NormalTok{\{}\OtherTok{leung2008}\NormalTok{,}
  \DataTypeTok{title}\NormalTok{ = \{Multicultural Experience Enhances Creativity: \{\{The\}\} When and How.\},}
  \DataTypeTok{volume}\NormalTok{ = \{63\},}
  \DataTypeTok{issn}\NormalTok{ = \{1935-990X(Electronic),0003-066X(Print)\},}
  \DataTypeTok{doi}\NormalTok{ = \{10.1037/0003-066X.63.3.169\},}
  \DataTypeTok{number}\NormalTok{ = \{3\},}
  \DataTypeTok{journaltitle}\NormalTok{ = \{American Psychologist\},}
  \DataTypeTok{date}\NormalTok{ = \{2008\},}
  \DataTypeTok{pages}\NormalTok{ = \{169-181\},}
  \DataTypeTok{keywords}\NormalTok{ = \{*Cognition,*Creativity,}
\NormalTok{    *Culture (Anthropological),}
\NormalTok{    *Experiences (Events),Multiculturalism\},}
  \DataTypeTok{author}\NormalTok{ = \{Leung, Angela Ka-yee and }
\NormalTok{    Maddux, William W. and }
\NormalTok{    Galinsky, Adam D. and Chiu, Chi-yue\}}
\NormalTok{\}}
\end{Highlighting}
\end{Shaded}

其中第一行的 \texttt{leung2008} 即為 citation key。透過 \texttt{@citekey}(\texttt{@leung2008})即可在文獻中插入 citation。匯出論文時,文末會自動產生引用的文獻。

\begin{itemize}
\tightlist
\item
  \texttt{Some\ text\ {[}@citekey{]}.}

  \begin{itemize}
  \tightlist
  \item
    Some text (Leung, Maddux, Galinsky, \& Chiu, \protect\hyperlink{ref-leung2008}{2008}).
  \end{itemize}
\item
  \texttt{@citekey\ Some\ text}

  \begin{itemize}
  \tightlist
  \item
    Leung et al. (\protect\hyperlink{ref-leung2008}{2008}) Some text
  \end{itemize}
\item
  \texttt{@citekey\ {[}p.\ 20{]}\ Some\ text.}

  \begin{itemize}
  \tightlist
  \item
    Leung et al. (\protect\hyperlink{ref-leung2008}{2008}, p. 20) Some text.
  \end{itemize}
\item
  \texttt{Some\ text\ {[}-@citekey{]}.}

  \begin{itemize}
  \tightlist
  \item
    Some text (\protect\hyperlink{ref-leung2008}{2008})
  \end{itemize}
\item
  \texttt{Some\ text\ {[}@citekey1;\ @citekey2{]}.}

  \begin{itemize}
  \tightlist
  \item
    Some text (Leung et al., \protect\hyperlink{ref-leung2008}{2008}; 黃宣範, \protect\hyperlink{ref-huangxuanfan1993}{1993}).
  \end{itemize}
\item
  Prefix \& Suffix

  \begin{itemize}
  \tightlist
  \item
    \texttt{Text\ {[}see\ @citekey1\ pp.45;\ also,\ @citekey2\ ch.\ 2{]}.}
  \item
    Text (see Leung et al., \protect\hyperlink{ref-leung2008}{2008}, p. 45; also, 黃宣範, \protect\hyperlink{ref-huangxuanfan1993}{1993} ch.~2).
  \end{itemize}
\end{itemize}

\hypertarget{ref-manager}{%
\subsection{書目管理軟體}\label{ref-manager}}

這裡建議使用 Zotero 加上 \href{https://retorque.re/zotero-better-bibtex/}{Better BibTeX} 擴充功能。\texttt{citr} 對 Zotero 有額外的支持,且 \textbf{Zotero 能夠控制 citation key 的格式}(例如,last name + year),但其它書目管理軟體如 Endnote 產生的 citation key 難以讀懂且無法更改格式。

\hypertarget{multi-lang-cite}{%
\subsection{多語言文獻引用}\label{multi-lang-cite}}

透過 csl 排版引用格式,只能支援單一語言。例如,若將英文格式套用到中文文獻,中文文獻就會出現英文的半形逗點和句點。

\hypertarget{cheatsheet}{%
\chapter{語法 Cheatsheet}\label{cheatsheet}}

下文許多內容直接自 \href{https://bookdown.org/yihui/bookdown/markdown-extensions-by-bookdown.html}{Bookdown} 擷取。想完整了解者,請閱讀 Bookdown 中的內容。

\hypertarget{chapter-cross-ref}{%
\section{章節連結}\label{chapter-cross-ref}}

\hypertarget{define-anchor}{%
\subsection{錨點定義}\label{define-anchor}}

在內文中,可以使用特殊語法建立前往其它章節的連結,但首先每個章節需要有\textbf{錨點}。若章節標題由英數組成,例如 \texttt{\#\#\ Experimental\ Design}, 則錨點會自動被定義成 \texttt{experimental-design}。若\textbf{標題含有中文},例如 \texttt{\#\#\ 實驗設計} 則需\textbf{自行定義錨點}:\texttt{\#\#\ 實驗設計\ \{\#exp-design\}}。

\subsection{內文連結}

要在內文建立連至其它章節的連結,需使用 \texttt{\textbackslash{}@ref(anchor)} 的語法,例如,

\begin{itemize}
\tightlist
\item
  連結至 \texttt{\#\#\ Experimental\ Design}

  \begin{itemize}
  \tightlist
  \item
    \texttt{\textbackslash{}@ref(experimental-design)}
  \end{itemize}
\item
  連結至 \texttt{\#\#\ 實驗設計\ \{\#exp-design\}}

  \begin{itemize}
  \tightlist
  \item
    \texttt{\textbackslash{}@ref(exp-design)}
  \end{itemize}
\end{itemize}

\hypertarget{math}{%
\section{數學}\label{math}}

\hypertarget{unnumbered-equations}{%
\subsection{Unnumbered Equations}\label{unnumbered-equations}}

\begin{Shaded}
\begin{Highlighting}[]
\KeywordTok{\textbackslash{}begin}\NormalTok{\{}\ExtensionTok{equation*}\NormalTok{\}}\SpecialStringTok{ }
\SpecialCharTok{\textbackslash{}frac}\SpecialStringTok{\{d\}\{dx\}}\SpecialCharTok{\textbackslash{}left}\SpecialStringTok{( }\SpecialCharTok{\textbackslash{}int}\SpecialStringTok{_\{a\}^\{x\} f(u)}\SpecialCharTok{\textbackslash{},}\SpecialStringTok{du}\SpecialCharTok{\textbackslash{}right}\SpecialStringTok{)=f(x)}
\KeywordTok{\textbackslash{}end}\NormalTok{\{}\ExtensionTok{equation*}\NormalTok{\} }
\end{Highlighting}
\end{Shaded}

\begin{equation*} 
\frac{d}{dx}\left( \int_{a}^{x} f(u)\,du\right)=f(x)
\end{equation*}

\hypertarget{numbered-equations}{%
\subsection{Numbered Equations}\label{numbered-equations}}

\begin{Shaded}
\begin{Highlighting}[]
\KeywordTok{\textbackslash{}begin}\NormalTok{\{}\ExtensionTok{equation}\NormalTok{\}}\SpecialStringTok{ }
\SpecialStringTok{  f}\SpecialCharTok{\textbackslash{}left}\SpecialStringTok{(k}\SpecialCharTok{\textbackslash{}right}\SpecialStringTok{) = }\SpecialCharTok{\textbackslash{}binom}\SpecialStringTok{\{n\}\{k\} p^k}\SpecialCharTok{\textbackslash{}left}\SpecialStringTok{(1-p}\SpecialCharTok{\textbackslash{}right}\SpecialStringTok{)^\{n-k\}}
\SpecialStringTok{  (}\SpecialCharTok{\textbackslash{}#}\SpecialStringTok{eq:bino)}
\KeywordTok{\textbackslash{}end}\NormalTok{\{}\ExtensionTok{equation}\NormalTok{\} }

\NormalTok{式 }\FunctionTok{\textbackslash{}@ref}\NormalTok{(eq:bino)}
\end{Highlighting}
\end{Shaded}

\begin{equation} 
  f\left(k\right) = \binom{n}{k} p^k\left(1-p\right)^{n-k}
  \label{eq:binom}
\end{equation}

式 \eqref{eq:binom}

\hypertarget{multi-line-aligned-equations}{%
\subsection{Multi-line Aligned Equations}\label{multi-line-aligned-equations}}

\begin{Shaded}
\begin{Highlighting}[]
\KeywordTok{\textbackslash{}begin}\NormalTok{\{}\ExtensionTok{equation}\NormalTok{\}}\SpecialStringTok{ }
\KeywordTok{\textbackslash{}begin}\NormalTok{\{}\ExtensionTok{split}\NormalTok{\}}
\SpecialCharTok{\textbackslash{}mathrm}\SpecialStringTok{\{Var\}(}\SpecialCharTok{\textbackslash{}hat}\SpecialStringTok{\{}\SpecialCharTok{\textbackslash{}beta}\SpecialStringTok{\}) & =}\SpecialCharTok{\textbackslash{}mathrm}\SpecialStringTok{\{Var\}((X'X)^\{-1\}X'y)}\SpecialCharTok{\textbackslash{}\textbackslash{}}
\SpecialStringTok{ & =(X'X)^\{-1\}X'}\SpecialCharTok{\textbackslash{}mathrm}\SpecialStringTok{\{Var\}(y)((X'X)^\{-1\}X')'}\SpecialCharTok{\textbackslash{}\textbackslash{}}
\SpecialStringTok{ & =(X'X)^\{-1\}X'}\SpecialCharTok{\textbackslash{}mathrm}\SpecialStringTok{\{Var\}(y)X(X'X)^\{-1\}}\SpecialCharTok{\textbackslash{}\textbackslash{}}
\SpecialStringTok{ & =(X'X)^\{-1\}X'}\SpecialCharTok{\textbackslash{}sigma}\SpecialStringTok{^\{2\}IX(X'X)^\{-1\}}\SpecialCharTok{\textbackslash{}\textbackslash{}}
\SpecialStringTok{ & =(X'X)^\{-1\}}\SpecialCharTok{\textbackslash{}sigma}\SpecialStringTok{^\{2\}}
\KeywordTok{\textbackslash{}end}\NormalTok{\{}\SpecialStringTok{split\}}
\SpecialStringTok{(}\SpecialCharTok{\textbackslash{}#}\SpecialStringTok{eq:var-beta)}
\KeywordTok{\textbackslash{}end}\NormalTok{\{}\ExtensionTok{equation}\NormalTok{\}}

\NormalTok{詳見公式 }\FunctionTok{\textbackslash{}@ref}\NormalTok{(eq:var-beta)}
\end{Highlighting}
\end{Shaded}

\begin{equation} 
\begin{split}
\mathrm{Var}(\hat{\beta}) & =\mathrm{Var}((X'X)^{-1}X'y)\\
 & =(X'X)^{-1}X'\mathrm{Var}(y)((X'X)^{-1}X')'\\
 & =(X'X)^{-1}X'\mathrm{Var}(y)X(X'X)^{-1}\\
 & =(X'X)^{-1}X'\sigma^{2}IX(X'X)^{-1}\\
 & =(X'X)^{-1}\sigma^{2}
\end{split}
\label{eq:var-bet}
\end{equation}

詳見公式 \eqref{eq:var-bet}

\hypertarget{theorem-proof}{%
\subsection{定理與證明}\label{theorem-proof}}

\begin{Shaded}
\begin{Highlighting}[]
\BaseNTok{```\{theorem, thm-label, name="Pythagorean theorem"\}}
\BaseNTok{For a right triangle, if $c$ denotes the length of the hypotenuse}
\BaseNTok{and $a$ and $b$ denote the lengths of the other two sides, we have}

\BaseNTok{$$a^2 + b^2 = c^2$$}
\BaseNTok{```}

\NormalTok{詳見定理 \textbackslash{}@ref(thm:thm-label)}
\end{Highlighting}
\end{Shaded}

詳見定理 \ref{thm:pyth}

\hypertarget{environment}{%
\subsection{環境}\label{environment}}

除了 \texttt{theorem} 之外,還可以使用其它 bookdown 提供的環境。例如,將 code chunk 中 \texttt{theorem} 換成 \texttt{proof}, \texttt{remark}, \texttt{solution}\footnote{\texttt{theorem}、\texttt{proof}、\texttt{remark} 這三者無法被 cross-reference。}。

表 \ref{tab:theorem-envs} 呈現 bookdown 中的\href{https://bookdown.org/yihui/bookdown/markdown-extensions-by-bookdown.html\#tab:theorem-envs}{引用環境}。



\begin{table}[t]

\caption{\label{tab:theorem-envs}Theorem environments in \textbf{bookdown}.}
\centering
\begin{tabular}{lll}
\toprule
Environment & Printed Name & Label Prefix\\
\midrule
theorem & Theorem & thm\\
lemma & Lemma & lem\\
corollary & Corollary & cor\\
proposition & Proposition & prp\\
conjecture & Conjecture & cnj\\
\addlinespace
definition & Definition & def\\
example & Example & exm\\
exercise & Exercise & exr\\
\bottomrule
\end{tabular}
\end{table}

\hypertarget{figure-referencing}{%
\section{Figure Referencing}\label{figure-referencing}}

\begin{Shaded}
\begin{Highlighting}[]
\AlertTok{![Test figure caption](img/watermark.png)}\NormalTok{\{#fig-with-pandoc width=50% \}}
\end{Highlighting}
\end{Shaded}

\begin{figure}
\hypertarget{fig-with-pandoc}{%
\centering
\includegraphics[width=0.5\textwidth,height=\textheight]{img/watermark.png}
\caption{Test figure caption}\label{fig-with-pandoc}
}
\end{figure}

\texttt{見圖\ \textbackslash{}@ref(fig:fig-with-pandoc)}產生:見圖 \ref{fig:fig-with-pandoc}

\hypertarget{figure-caption}{%
\subsection{Figure Caption}\label{figure-caption}}

對於比較複雜的 caption,可以使用 Text reference 的方式:

\begin{Shaded}
\begin{Highlighting}[]
\NormalTok{對於比較複雜的 caption,可以使用 Text reference 的方式:}

\NormalTok{(ref:dia) 插入**引用資料** [@kassin2017] 的 Figure Caption.}

\AlertTok{![(ref:dia)](img/watermark.png)}\NormalTok{\{#fig-with-pandoc2 width=50% \}}
\end{Highlighting}
\end{Shaded}

\hypertarget{table-referencing}{%
\section{Table Referencing}\label{table-referencing}}



\begin{longtable}[]{@{}rlcl@{}}
\caption{This is table caption.}\tabularnewline
\toprule
Right & Left & Center & Default\tabularnewline
\midrule
\endfirsthead
\toprule
Right & Left & Center & Default\tabularnewline
\midrule
\endhead
12 & 12 & 12 & 12\tabularnewline
123 & 123 & 123 & 123\tabularnewline
1 & 1 & 1 & 1\tabularnewline
\bottomrule
\end{longtable}

\renewcommand{\href}{\oldhref}

\hypertarget{references}{%
\chapter*{參考資料}\label{references}}
\addcontentsline{toc}{chapter}{參考資料}

\hypertarget{refs}{}
\leavevmode\hypertarget{ref-leung2008}{}%
Leung, A. K.-y., Maddux, W. W., Galinsky, A. D., \& Chiu, C.-y. (2008). Multicultural experience enhances creativity: The when and how. \emph{American Psychologist}, \emph{63}(3), 169--181. \url{https://doi.org/10.1037/0003-066X.63.3.169}

\leavevmode\hypertarget{ref-huangxuanfan1993}{}%
黃宣範. (1993). \emph{語言、社會與族群意識: 臺灣語言社會學的研究}. 臺北市: 文鶴. Retrieved from \url{http://tulips.ntu.edu.tw:1081/record=b1285025*cht}







\end{document}
