  \documentclass[oneside]{book}
%\documentclass[12pt,]{book}

\usepackage{lmodern}
\usepackage{setspace}
\setstretch{2}
\usepackage{amssymb,amsmath}
\usepackage{ifxetex,ifluatex}
\usepackage{fixltx2e} % provides \textsubscript
\ifnum 0\ifxetex 1\fi\ifluatex 1\fi=0 % if pdftex
  \usepackage[T1]{fontenc}
  \usepackage[utf8]{inputenc}
\else % if luatex or xelatex
  \ifxetex
    \usepackage{xltxtra,xunicode}
  \else
    \usepackage{fontspec}
  \fi
  \defaultfontfeatures{Ligatures=TeX,Scale=MatchLowercase}


\fi
% use upquote if available, for straight quotes in verbatim environments
\IfFileExists{upquote.sty}{\usepackage{upquote}}{}
% use microtype if available
\IfFileExists{microtype.sty}{%
\usepackage{microtype}
\UseMicrotypeSet[protrusion]{basicmath} % disable protrusion for tt fonts
}{}
\usepackage[a4paper, left=1.18in, right=1.18in, top=1.18in, bottom=0.787in]{geometry}
\usepackage[unicode=true]{hyperref}
\hypersetup{
            pdftitle={臺灣大學論文 Bookdown 模板},
            pdfauthor={廖永賦},
            pdfborder={0 0 0},
            breaklinks=true}
\urlstyle{same}  % don't use monospace font for urls
\usepackage{color}
\usepackage{fancyvrb}
\newcommand{\VerbBar}{|}
\newcommand{\VERB}{\Verb[commandchars=\\\{\}]}
\DefineVerbatimEnvironment{Highlighting}{Verbatim}{commandchars=\\\{\}}
% Add ',fontsize=\small' for more characters per line
\usepackage{framed}
\definecolor{shadecolor}{RGB}{248,248,248}
\newenvironment{Shaded}{\begin{snugshade}}{\end{snugshade}}
\newcommand{\KeywordTok}[1]{\textcolor[rgb]{0.13,0.29,0.53}{\textbf{#1}}}
\newcommand{\DataTypeTok}[1]{\textcolor[rgb]{0.13,0.29,0.53}{#1}}
\newcommand{\DecValTok}[1]{\textcolor[rgb]{0.00,0.00,0.81}{#1}}
\newcommand{\BaseNTok}[1]{\textcolor[rgb]{0.00,0.00,0.81}{#1}}
\newcommand{\FloatTok}[1]{\textcolor[rgb]{0.00,0.00,0.81}{#1}}
\newcommand{\ConstantTok}[1]{\textcolor[rgb]{0.00,0.00,0.00}{#1}}
\newcommand{\CharTok}[1]{\textcolor[rgb]{0.31,0.60,0.02}{#1}}
\newcommand{\SpecialCharTok}[1]{\textcolor[rgb]{0.00,0.00,0.00}{#1}}
\newcommand{\StringTok}[1]{\textcolor[rgb]{0.31,0.60,0.02}{#1}}
\newcommand{\VerbatimStringTok}[1]{\textcolor[rgb]{0.31,0.60,0.02}{#1}}
\newcommand{\SpecialStringTok}[1]{\textcolor[rgb]{0.31,0.60,0.02}{#1}}
\newcommand{\ImportTok}[1]{#1}
\newcommand{\CommentTok}[1]{\textcolor[rgb]{0.56,0.35,0.01}{\textit{#1}}}
\newcommand{\DocumentationTok}[1]{\textcolor[rgb]{0.56,0.35,0.01}{\textbf{\textit{#1}}}}
\newcommand{\AnnotationTok}[1]{\textcolor[rgb]{0.56,0.35,0.01}{\textbf{\textit{#1}}}}
\newcommand{\CommentVarTok}[1]{\textcolor[rgb]{0.56,0.35,0.01}{\textbf{\textit{#1}}}}
\newcommand{\OtherTok}[1]{\textcolor[rgb]{0.56,0.35,0.01}{#1}}
\newcommand{\FunctionTok}[1]{\textcolor[rgb]{0.00,0.00,0.00}{#1}}
\newcommand{\VariableTok}[1]{\textcolor[rgb]{0.00,0.00,0.00}{#1}}
\newcommand{\ControlFlowTok}[1]{\textcolor[rgb]{0.13,0.29,0.53}{\textbf{#1}}}
\newcommand{\OperatorTok}[1]{\textcolor[rgb]{0.81,0.36,0.00}{\textbf{#1}}}
\newcommand{\BuiltInTok}[1]{#1}
\newcommand{\ExtensionTok}[1]{#1}
\newcommand{\PreprocessorTok}[1]{\textcolor[rgb]{0.56,0.35,0.01}{\textit{#1}}}
\newcommand{\AttributeTok}[1]{\textcolor[rgb]{0.77,0.63,0.00}{#1}}
\newcommand{\RegionMarkerTok}[1]{#1}
\newcommand{\InformationTok}[1]{\textcolor[rgb]{0.56,0.35,0.01}{\textbf{\textit{#1}}}}
\newcommand{\WarningTok}[1]{\textcolor[rgb]{0.56,0.35,0.01}{\textbf{\textit{#1}}}}
\newcommand{\AlertTok}[1]{\textcolor[rgb]{0.94,0.16,0.16}{#1}}
\newcommand{\ErrorTok}[1]{\textcolor[rgb]{0.64,0.00,0.00}{\textbf{#1}}}
\newcommand{\NormalTok}[1]{#1}
\usepackage{longtable,booktabs}
% Fix footnotes in tables (requires footnote package)
\IfFileExists{footnote.sty}{\usepackage{footnote}\makesavenoteenv{long table}}{}
\let\oldhref=\href
% Make links footnotes instead of hotlinks:
\renewcommand{\href}[2]{#2\footnote{\url{#1}}}
\IfFileExists{parskip.sty}{%
\usepackage{parskip}
}{% else
\setlength{\parindent}{0pt}
\setlength{\parskip}{6pt plus 2pt minus 1pt}
}
\setlength{\emergencystretch}{3em}  % prevent overfull lines
\providecommand{\tightlist}{%
  \setlength{\itemsep}{0pt}\setlength{\parskip}{0pt}}
\setcounter{secnumdepth}{5}

% set default figure placement to htbp
\makeatletter
\def\fps@figure{htbp}
\makeatother

\usepackage{pdfpages}
\usepackage{titlesec}
\usepackage{titletoc}
\usepackage{xCJKnumb}
\usepackage{booktabs}

%中文自動換行
\XeTeXlinebreaklocale "zh"
%文字的彈性間距
\XeTeXlinebreakskip = 0pt plus 1pt


\titleformat{\chapter}{\centering\Huge\bfseries}{第\,\xCJKnumber{\thechapter}\,章}{1em} {}
\titlespacing{\chapter}{0cm}{-1.3cm}{1em} 

\renewcommand{\figurename}{圖}
\renewcommand{\tablename}{表}
\renewcommand{\contentsname}{目錄}
\renewcommand{\listfigurename}{圖目錄}
\renewcommand{\listtablename}{表目錄}
%\renewcommand{\bibname}{參考資料}

\titlecontents{chapter}[0em]
{}{\makebox[4.1em][l]
{第\xCJKnumber{\thecontentslabel}章}}{}
{\titlerule*[0.7pc]{.}\contentspage}

\title{臺灣大學論文 Bookdown 模板}
\author{廖永賦}
\date{October 17, 2018}


\usepackage{fontspec}
%使用xeCJK,其他的還有CJK或是xCJK
\usepackage{xeCJK}
\usepackage{bm}

% Set the default fonts
  \setmainfont{Liberation Serif} % Times New Roman  %Liberation Serif

                        \setCJKmainfont[AutoFakeBold=2.5, AutoFakeSlant=.3]{AR PL KaitiM Big5} %標楷體 %AR PL KaitiM Big5
            %\setCJKmainfont[AutoFakeBold=1,AutoFakeSlant=.4]{BiauKai}
            


% IPA support (Works with linguisticsdown)
  \newfontfamily\ipa{Doulos SIL} % Font for IPA symbols
  \DeclareTextFontCommand{\ipatext}{\ipa}


\usepackage{xcolor}
\usepackage{transparent}
\usepackage[printwatermark]{xwatermark}

  \newwatermark*[allpages,xpos=6.1725cm,ypos=10.5225cm,scale=0.5]{\includegraphics{watermark.pdf}}


\begin{document}


\includepdf[pages={1}, scale=1]{front_matter/front_matter.pdf}

\clearpage
\pagenumbering{roman}

\phantomsection
\addcontentsline{toc}{chapter}{口試委員會審定書}
\includepdf[pages={1}, scale=1]{front_matter/certification.pdf}

\phantomsection
\addcontentsline{toc}{chapter}{誌謝}
\includepdf[pages={3}, scale=1]{front_matter/front_matter.pdf}


\phantomsection
\addcontentsline{toc}{chapter}{中文摘要}
\includepdf[pages={4}, scale=1]{front_matter/front_matter.pdf}

\phantomsection
\addcontentsline{toc}{chapter}{英文摘要}
\includepdf[pages={5}, scale=1]{front_matter/front_matter.pdf}

\clearpage
\pagenumbering{arabic}

{
\setcounter{tocdepth}{1}
\tableofcontents
}

\newpage

\listoftables
\newpage
\listoffigures
\newpage

% Set independent linestretch for code chunks
\let\oldShaded=\Shaded
\let\endoldShaded=\endShaded
\renewenvironment{Shaded}{
      \begin{spacing}{1.5}\begin{oldShaded}
    }
  {
  \end{oldShaded}
  \end{spacing}
  }


\chapter{Introduction}\label{intro}

\ipatext{ä} is the IPA symbol for ㄚ in ㄅㄆㄇ。

\textbf{中文}中文\emph{中文}\\
可以換行

新段落

By default, bookdown merges all Rmd files by the order of filenames,
e.g., \texttt{01-intro.Rmd} will appear before
\texttt{02-literature.Rmd}. Filenames that start with an underscore
\texttt{\_} are skipped. \texttt{index.Rmd} will always be treated as
the first file.

Stanford (\protect\hyperlink{ref-stanford2008}{2008})
英文參考文獻。中文參考文獻 (黃宣範,
\protect\hyperlink{ref-huangxuanfan1993}{1993})

Alternatively, set the order in \texttt{\_bookdown.yml}:

\begin{Shaded}
\begin{Highlighting}[]
\FunctionTok{rmd_files:}
  \FunctionTok{html:}\AttributeTok{ }\KeywordTok{[}\StringTok{"index.Rmd"}\KeywordTok{,} \StringTok{"abstract.Rmd"}\KeywordTok{,} \StringTok{"intro.Rmd"}\KeywordTok{]}
  \FunctionTok{latex:}\AttributeTok{ }\KeywordTok{[}\StringTok{"abstract.Rmd"}\KeywordTok{,} \StringTok{"intro.Rmd"}\KeywordTok{]}
\end{Highlighting}
\end{Shaded}

\begin{itemize}
\tightlist
\item
  Benefits
\item
  Quick Start

  \begin{itemize}
  \tightlist
  \item
    General Usage

    \begin{itemize}
    \tightlist
    \item
      Front matter
    \item
      Content
    \end{itemize}
  \item
    Write Thesis in English (Write in Eng)
  \end{itemize}
\item
  Bookdown Demo

  \begin{enumerate}
  \def\labelenumi{\alph{enumi})}
  \tightlist
  \item
    Markdown quick guide

    \begin{itemize}
    \tightlist
    \item
      no closing or opening quotation marks
    \end{itemize}
  \item
    Citation
  \item
    Cross References
  \end{enumerate}

  \begin{itemize}
  \tightlist
  \item
    數學公式
  \item
    indexing
  \end{itemize}
\item
  Harnessing the Power of R Community

  \begin{enumerate}
  \def\labelenumi{\Alph{enumi})}
  \tightlist
  \item
    Facilitating Packages

    \begin{itemize}
    \tightlist
    \item
      citr
    \end{itemize}
  \item
    特定領域

    \begin{itemize}
    \tightlist
    \item
      linguisticsdown
    \end{itemize}
  \end{enumerate}
\item
  Resources

  \begin{itemize}
  \tightlist
  \item
    R Markdown Features
  \end{itemize}
\item
  Thanks and Help

  \begin{itemize}
  \tightlist
  \item
    GitHub issues
  \end{itemize}
\end{itemize}

\begin{Shaded}
\begin{Highlighting}[]
\NormalTok{├── index.Rmd              }\CommentTok{# Book Layout (font, watermark, biblio, ...)}
\NormalTok{├── 01-intro.Rmd           }\CommentTok{# Chapter 1 content}
\NormalTok{├── 02-literature.Rmd      }\CommentTok{# Chapter 2 content}
\NormalTok{├── 03-method.Rmd          }\CommentTok{# Chapter 3 content}
\NormalTok{├── 99-references.Rmd      }\CommentTok{# Don't need to edit}
\NormalTok{├── ref.bib                }\CommentTok{# References}
\NormalTok{├── cite-style.csl         }\CommentTok{# Citation style}
\NormalTok{├── watermark.pdf          }\CommentTok{# 臺大浮水印 (PDF 右上角)   }
\NormalTok{├── _bookdown.yml          }\CommentTok{# label names in html; Rmd files order}
\NormalTok{├── _output.yml            }\CommentTok{# preamble, pandoc args, cite-pkg}
\NormalTok{├── front_matter}
\NormalTok{│   ├── front_matter.rmd   }\CommentTok{# Edit}
\NormalTok{│   ├── certification.pdf  }\CommentTok{# Prepare scan file}
\NormalTok{│   └── front_matter.pdf   }\CommentTok{# Auto generated}
\NormalTok{└── latex}
\NormalTok{    ├── preamble-en.tex    }\CommentTok{# _output.yml}
\NormalTok{    └── preamble-zh.tex    }\CommentTok{# _output.yml, for Eng thesis}
\end{Highlighting}
\end{Shaded}

Test \href{http://pandoc.org/MANUAL.html\#variables-for-latex}{link as
note}.

\textbf{You} can label chapter and section titles using
\texttt{\{\#label\}} after them, e.g., we can reference Chapter
\ref{intro}. If you do not manually label them, there will be automatic
labels anyway, e.g., Chapter \ref{methods}.

Figures and tables with captions will be placed in \texttt{figure} and
\texttt{table} environments, respectively.

\begin{Shaded}
\begin{Highlighting}[]
\KeywordTok{par}\NormalTok{(}\DataTypeTok{mar =} \KeywordTok{c}\NormalTok{(}\DecValTok{4}\NormalTok{, }\DecValTok{4}\NormalTok{, .}\DecValTok{1}\NormalTok{, .}\DecValTok{1}\NormalTok{))}
\KeywordTok{plot}\NormalTok{(pressure, }\DataTypeTok{type =} \StringTok{'b'}\NormalTok{, }\DataTypeTok{pch =} \DecValTok{19}\NormalTok{)}
\end{Highlighting}
\end{Shaded}

\begin{figure}

{\centering \includegraphics[width=0.8\linewidth]{ntu-bookdown_files/figure-latex/nice-fig-1} 

}

\caption{Here is a nice figure!}\label{fig:nice-fig}
\end{figure}

Reference a figure by its code chunk label with the \texttt{fig:}
prefix, e.g., see Figure \ref{fig:nice-fig}. Similarly, you can
reference tables generated from \texttt{knitr::kable()}, e.g., see Table
\ref{tab:nice-tab}.

\begin{Shaded}
\begin{Highlighting}[]
\NormalTok{knitr}\OperatorTok{::}\KeywordTok{kable}\NormalTok{(}
  \KeywordTok{head}\NormalTok{(iris, }\DecValTok{20}\NormalTok{), }\DataTypeTok{caption =} \StringTok{'Here is a nice table!'}\NormalTok{,}
  \DataTypeTok{booktabs =} \OtherTok{TRUE}
\NormalTok{)}
\end{Highlighting}
\end{Shaded}

\begin{table}

\caption{\label{tab:nice-tab}Here is a nice table!}
\centering
\begin{tabular}[t]{rrrrl}
\toprule
Sepal.Length & Sepal.Width & Petal.Length & Petal.Width & Species\\
\midrule
5.1 & 3.5 & 1.4 & 0.2 & setosa\\
4.9 & 3.0 & 1.4 & 0.2 & setosa\\
4.7 & 3.2 & 1.3 & 0.2 & setosa\\
4.6 & 3.1 & 1.5 & 0.2 & setosa\\
5.0 & 3.6 & 1.4 & 0.2 & setosa\\
\addlinespace
5.4 & 3.9 & 1.7 & 0.4 & setosa\\
4.6 & 3.4 & 1.4 & 0.3 & setosa\\
5.0 & 3.4 & 1.5 & 0.2 & setosa\\
4.4 & 2.9 & 1.4 & 0.2 & setosa\\
4.9 & 3.1 & 1.5 & 0.1 & setosa\\
\addlinespace
5.4 & 3.7 & 1.5 & 0.2 & setosa\\
4.8 & 3.4 & 1.6 & 0.2 & setosa\\
4.8 & 3.0 & 1.4 & 0.1 & setosa\\
4.3 & 3.0 & 1.1 & 0.1 & setosa\\
5.8 & 4.0 & 1.2 & 0.2 & setosa\\
\addlinespace
5.7 & 4.4 & 1.5 & 0.4 & setosa\\
5.4 & 3.9 & 1.3 & 0.4 & setosa\\
5.1 & 3.5 & 1.4 & 0.3 & setosa\\
5.7 & 3.8 & 1.7 & 0.3 & setosa\\
5.1 & 3.8 & 1.5 & 0.3 & setosa\\
\bottomrule
\end{tabular}
\end{table}

\chapter{Literature}\label{literature}

Here is a review of existing methods.

\section{Level}\label{level}

sdfdf

\chapter{Methods}\label{methods}

We describe our methods in this chapter.

\renewcommand{\href}{\oldhref}

\chapter*{參考資料}\label{references}
\addcontentsline{toc}{chapter}{參考資料}

\hypertarget{refs}{}
\hypertarget{ref-stanford2008}{}
Stanford, J. N. (2008). Child dialect acquisition: New perspectives on
parent/peer influence1. \emph{Journal of Sociolinguistics},
\emph{12}(5), 567--596.
\url{https://doi.org/10.1111/j.1467-9841.2008.00383.x}

\hypertarget{ref-huangxuanfan1993}{}
黃宣範. (1993). \emph{語言、社會與族群意識: 臺灣語言社會學的研究}.
臺北市: 文鶴. Retrieved from
\url{http://tulips.ntu.edu.tw:1081/record=b1285025*cht}







\end{document}
